\documentclass{article}
\usepackage[
backend=biber,
style=alphabetic,
sorting=ynt
]{biblatex}
\addbibresource{bibliography.bib}
\usepackage{graphicx} % Required for inserting images
\usepackage[portuguese]{babel}
\usepackage{hyperref}
\title{IA Relatório - Projeto 1: Uso de inteligência artificial em visão computacional}
\author{Victor Seixas Locateli RA: 10364921}

\begin{document}

\maketitle
\newpage
\section{
Integrantes
}
\begin{itemize}
    \item Victor Seixas Locateli - RA: 10364921
\end{itemize}
    
\section{
Resumo
}

Este projeto tem como fim utilizar a inteligência artifical disponível na biblioteca OpenCV para a identificação de rostos. Pretende-se verificar a perfomance das classes disponíveis no OpenCV. A realização deste trabalho é feita em ambiente Linux, com a seguinte especificação do computador:
\begin{itemize}
    \item CPU: I7 7700K
    \item RAM: 16GB 
    \item GPU: NVIDIA 1070 OC
\end{itemize}
\section{Introdução}
\subsection{
Contextualização
}

A identificação e o uso exarcebado de inteligência artificial, por parte da população geral tem sido motivo de preocupação, principalmente por questão éticas e legais, onde o uso de uma ferramenta aparentemente inovadora e inofensiva pode ser utilizada para meios ilícitos.
\subsection{Justificativa}
Pretende-se demonstrar como é simples identificar um rosto com o OpenCV, isto é, demonstrar que é muito simples começar a utilizar essa ferramenta. Para substituição de um rosto, há ferramentas FOSS disponíveis, mas por questões legais quase todas foram discontinuadas. Essa facilidade mostra o quão perigoso pode ser o uso dessa ferramenta sem que instituições governamentais tomem alguma medida de punição cabível.
\subsection{Objetivo}
Pretende mostrar como é fácil utilizar a biblioteca OpenCV, mesmo que em uma linguagem complexa como C++. Como este projeto será usado na disciplina de \textbf{computação visual}, há um outro PDF neste repositório explicando como o código funciona.
\subsection{Framework: OpenCV}
Foi utilizado o Framework OpenCV para o processamento de computação visual com os filtros de deep learning disponíveis.
\section{Descrição do Problema}
O problema seria de indentificar rostos em imagens. Pretende-se mostrar como é fácil utilizar o framework OpenCV e com algumas linhas de código, já possui uma ferramente que consegue identificar rostos.
\subsubsection{Aspectos éticos}
Foi se disctuido na subseção Justificativa, sobre os aspectos éticos.
\subsection{Dataset}
\href{https://www.kaggle.com/datasets/kasikrit/att-database-of-faces} {Dataset de rostos da AT\&T}
\subsection{Resultados esperados}
Pretende-se idenfiticar os rostos e se possíveis classificar a idade e gênero, respectivamente, utilizando como referência o projeto da Webcam de \cite{cvbookpakt}
\printbibliography
\newpage
\end{document} 
